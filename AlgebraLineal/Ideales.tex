
\newpage
\section{Ideales, Homomorfismos y Anillos}
    \begin{definicion}
        Una función $\varphi: R \to R'$ es un homomorfismo si:

        \begin{enumerate}
            \item $\varphi(a) + \varphi(b) = \varphi(a + b)$
            \item $\varphi(a) \cdot \varphi(b) = \varphi(a \cdot b)$
        \end{enumerate}
    \end{definicion}

    \begin{definicion}
        Sea $\varphi: R \to R'$ homomorfismo de anillos, entonces:
        \begin{enumerate}
            \item $\varphi$ es monomorfismo si es inyectivo
            \item $\varphi$ es epimorfismo si es suprayectivo
            \item $\varphi$ es isomorfismo si es biyectivo
        \end{enumerate}
    \end{definicion}

    \begin{definicion}
        El nucleo de $\varphi$ es $\ker{\varphi} = \left\{ x \in R \mid \varphi(x) = 0 \right\}$
    \end{definicion}

    \begin{proposicion}
        Sea $\varphi:R \to R'$, es un homomorfismo de anillos, entonces:

        \begin{enumerate}
            \item $\ker{\varphi}$ es un subgrupo aditivo
            \item $rk, kr \in \ker{\varphi} \quad \forall k \in \ker{\varphi} \quad r \in R$
        \end{enumerate}
    \end{proposicion}

    \begin{proof}
        Sea $k \in \ker{\varphi}$ y $r \in R$

        \begin{eqnarray*}
            \varphi(rk) & = & \varphi(r) \cdot \varphi(k) \\
            & = & \varphi(r) \cdot 0 = 0
        \end{eqnarray*}

        \begin{eqnarray*}
            \varphi(kr) & = & \varphi(k) \cdot \varphi(r) \\
            & = & 0 \cdot \varphi(r) = 0
        \end{eqnarray*}

        \begin{equation*}
            \therefore kr, rk \in \ker{\varphi}
        \end{equation*}
    \end{proof}

    \begin{definicion}
        Sea $R$ un anillo e $I$ un subconjunto de $R$, se dice que es un ideal de $R$ si:

        \begin{enumerate}[i)]
            \item $I$ es un subgrupo aditivo de $R$
            \item Dados $r \in R$ y $a \in I$, tenemos que $ra \in I$ y $ar \in I$
        \end{enumerate}

        a esto se le conoce como propiedad de absorción.
    \end{definicion}
