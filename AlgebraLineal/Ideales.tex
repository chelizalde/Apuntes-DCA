
\newpage
\section{Ideales, Homomorfismos y Anillos}
    \begin{definicion}
        Una función $\varphi: R \to R'$ es un homomorfismo si:

        \begin{enumerate}
            \item $\varphi(a) + \varphi(b) = \varphi(a + b)$
            \item $\varphi(a) \cdot \varphi(b) = \varphi(a \cdot b)$
        \end{enumerate}
    \end{definicion}

    \begin{definicion}
        Sea $\varphi: R \to R'$ homomorfismo de anillos, entonces:
        \begin{enumerate}
            \item $\varphi$ es monomorfismo si es inyectivo
            \item $\varphi$ es epimorfismo si es suprayectivo
            \item $\varphi$ es isomorfismo si es biyectivo
        \end{enumerate}
    \end{definicion}

    \begin{definicion}
        El nucleo de $\varphi$ es $\ker{\varphi} = \left\{ x \in R \mid \varphi(x) = 0 \right\}$
    \end{definicion}

    \begin{proposicion}
        Sea $\varphi:R \to R'$, es un homomorfismo de anillos, entonces:

        \begin{enumerate}
            \item $\ker{\varphi}$ es un subgrupo aditivo
            \item $rk, kr \in \ker{\varphi} \quad \forall k \in \ker{\varphi} \quad r \in R$
        \end{enumerate}
    \end{proposicion}

    \begin{proof}
        Sea $k \in \ker{\varphi}$ y $r \in R$

        \begin{eqnarray*}
            \varphi(rk) & = & \varphi(r) \cdot \varphi(k) \\
            & = & \varphi(r) \cdot 0 = 0
        \end{eqnarray*}

        \begin{eqnarray*}
            \varphi(kr) & = & \varphi(k) \cdot \varphi(r) \\
            & = & 0 \cdot \varphi(r) = 0
        \end{eqnarray*}

        \begin{equation*}
            \therefore kr, rk \in \ker{\varphi}
        \end{equation*}
    \end{proof}

    \begin{definicion}
        Sea $R$ un anillo e $I$ un subconjunto de $R$, se dice que es un ideal de $R$ si:

        \begin{enumerate}[i)]
            \item $I$ es un subgrupo aditivo de $R$
            \item Dados $r \in R$ y $a \in I$, tenemos que $ra \in I$ y $ar \in I$
        \end{enumerate}

        a esto se le conoce como propiedad de absorción.
    \end{definicion}

    \begin{corolario}
        Si $\varphi: R \to R'$ es un homomorfismo, entonces $k = \ker{\varphi}$ es un ideal de $R$.
    \end{corolario}

    \begin{definicion}
        Sea $R$ anillo e $I$ un ideal de $R$, entonces $\sfrac{R}{I}$ (anillo cociente) es un grupo con la suma de clases de equivalencia.

        \begin{equation}
            (a + I) + (b + I) = (a + b) + I \quad \forall a, b \in R
        \end{equation}
    \end{definicion}

    \begin{definicion}
        Definimos el producto como:

        \begin{equation}
            (a + I)(b + I) = ab + I \quad \forall a, b \in R
        \end{equation}
    \end{definicion}

    \begin{observacion}
        Sea $R$ un anillo, tenemos que $\{ 0 \}$ y $R$ son ideales de $R$ y triviales.
    \end{observacion}

    \begin{definicion}
        Si $I$ es un ideal de $R$ e $I \ne R$, decimos que $I$ es un ideal propio.
    \end{definicion}

    \begin{observacion}
        Sea $R$ un anillo con identidad e $I$ ideal de $R$ tal que $1 \in I$, entonces $I = R$.
    \end{observacion}

    \begin{proof}
        Por definicion $I \subseteq R$, por lo que solo resta demostrar que $R \subset I$.

        Sea $a \in R$, entonces:

        \begin{equation*}
            a = a \cdot I \quad 1 \in I \quad a \in R
        \end{equation*}

        por lo tanto $a \in I$, $\therefore R \subseteq I$.
    \end{proof}

    \begin{definicion}
        Sea $R$ anillo conmutativo con identidad. Un ideal principal, es un ideal de la forma $(a)$, para algun $a \in R$.
    \end{definicion}

    \begin{definicion}
        Sea $R$ un dominio enterio, diremos que $R$ es un dominio de ideales principales, si todos los ideales de $R$ son principales.
    \end{definicion}

    \begin{observacion}
        Sean $I, J$ ideales del anillo $R$, si definimos:

        \begin{equation}
            I + J = \left\{ a + b \mid a \in I, b \in J \right\}
        \end{equation}

        Verificar que:
        
        \begin{enumerate}
            \item $I + J$ es un ideal
            \item $I \cap J$ es un ideal
            \item $IJ := \left\{ \sum_{i=1}^n a_i b_i \mid n \in \mathbbm{N}, a \in I, b \in J \right\}$ es un ideal.
        \end{enumerate}

        \todo{Verificación asignada a tarea.}
    \end{observacion}

    \begin{observacion}
        Tenemos que $\sfrac{R}{I}$ es un anillo con la suma y el producto correspondientes:

        \begin{eqnarray}
            (a + I) + (b + I) & := & (a + b) + I \\
            (a + I) \cdot (b + I) & := & (a \cdot b) + I
        \end{eqnarray}

        entonces la función:

        \begin{eqnarray*}
            \varphi & := & R \to \sfrac{R}{I} \\
             & & a \to a + I
        \end{eqnarray*}

        es un epimorfismo
        \todo{Verificación asignada a tarea.}
    \end{observacion}
