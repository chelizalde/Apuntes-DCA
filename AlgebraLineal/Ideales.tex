
\newpage
\section{Ideales, Homomorfismos y Anillos}
    \begin{definicion}
        Una función $\varphi: R \to R'$ es un homomorfismo si:

        \begin{enumerate}
            \item $\varphi(a) + \varphi(b) = \varphi(a + b)$
            \item $\varphi(a) \cdot \varphi(b) = \varphi(a \cdot b)$
        \end{enumerate}
    \end{definicion}

    \begin{definicion}
        Sea $\varphi: R \to R'$ homomorfismo de anillos, entonces:
        \begin{enumerate}
            \item $\varphi$ es monomorfismo si es inyectivo
            \item $\varphi$ es epimorfismo si es suprayectivo
            \item $\varphi$ es isomorfismo si es biyectivo
        \end{enumerate}
    \end{definicion}

    \begin{definicion}
        El nucleo de $\varphi$ es $\ker{\varphi} = \left\{ x \in R \mid \varphi(x) = 0 \right\}$
    \end{definicion}

    \begin{proposicion}
        Sea $\varphi:R \to R'$, es un homomorfismo de anillos, entonces:

        \begin{enumerate}
            \item $\ker{\varphi}$ es un subgrupo aditivo
            \item $rk, kr \in \ker{\varphi} \quad \forall k \in \ker{\varphi} \quad r \in R$
        \end{enumerate}
    \end{proposicion}

    \begin{proof}
        Sea $k \in \ker{\varphi}$ y $r \in R$

        \begin{eqnarray*}
            \varphi(rk) & = & \varphi(r) \cdot \varphi(k) \\
            & = & \varphi(r) \cdot 0 = 0
        \end{eqnarray*}

        \begin{eqnarray*}
            \varphi(kr) & = & \varphi(k) \cdot \varphi(r) \\
            & = & 0 \cdot \varphi(r) = 0
        \end{eqnarray*}

        \begin{equation*}
            \therefore kr, rk \in \ker{\varphi}
        \end{equation*}
    \end{proof}

    \begin{definicion}
        Sea $R$ un anillo e $I$ un subconjunto de $R$, se dice que es un ideal de $R$ si:

        \begin{enumerate}[i)]
            \item $I$ es un subgrupo aditivo de $R$
            \item Dados $r \in R$ y $a \in I$, tenemos que $ra \in I$ y $ar \in I$
        \end{enumerate}

        a esto se le conoce como propiedad de absorción.
    \end{definicion}

    \begin{corolario}
        Si $\varphi: R \to R'$ es un homomorfismo, entonces $k = \ker{\varphi}$ es un ideal de $R$.
    \end{corolario}

    \begin{definicion}
        Sea $R$ anillo e $I$ un ideal de $R$, entonces $\sfrac{R}{I}$ (anillo cociente) es un grupo con la suma de clases de equivalencia.

        \begin{equation}
            (a + I) + (b + I) = (a + b) + I \quad \forall a, b \in R
        \end{equation}
    \end{definicion}

    \begin{definicion}
        Definimos el producto como:

        \begin{equation}
            (a + I)(b + I) = ab + I \quad \forall a, b \in R
        \end{equation}
    \end{definicion}

    \begin{observacion}
        Sea $R$ un anillo, tenemos que $\{ 0 \}$ y $R$ son ideales de $R$ y triviales.
    \end{observacion}

    \begin{definicion}
        Si $I$ es un ideal de $R$ e $I \ne R$, decimos que $I$ es un ideal propio.
    \end{definicion}

    \begin{observacion}
        Sea $R$ un anillo con identidad e $I$ ideal de $R$ tal que $1 \in I$, entonces $I = R$.
    \end{observacion}

    \begin{proof}
        Por definicion $I \subseteq R$, por lo que solo resta demostrar que $R \subset I$.

        Sea $a \in R$, entonces:

        \begin{equation*}
            a = a \cdot I \quad 1 \in I \quad a \in R
        \end{equation*}

        por lo tanto $a \in I$, $\therefore R \subseteq I$.
    \end{proof}

    \begin{definicion}
        Sea $R$ anillo conmutativo con identidad. Un ideal principal, es un ideal de la forma $(a)$, para algun $a \in R$.
    \end{definicion}

    \begin{definicion}
        Sea $R$ un dominio enterio, diremos que $R$ es un dominio de ideales principales, si todos los ideales de $R$ son principales.
    \end{definicion}

    \begin{observacion}
        Sean $I, J$ ideales del anillo $R$, si definimos:

        \begin{equation}
            I + J = \left\{ a + b \mid a \in I, b \in J \right\}
        \end{equation}

        Verificar que:

        \begin{enumerate}
            \item $I + J$ es un ideal
            \item $I \cap J$ es un ideal
            \item $IJ := \left\{ \sum_{i=1}^n a_i b_i \mid n \in \mathbbm{N}, a \in I, b \in J \right\}$ es un ideal.
        \end{enumerate}

        \todo{Verificación asignada a tarea.}
    \end{observacion}

    \begin{observacion}
        Tenemos que $\sfrac{R}{I}$ es un anillo con la suma y el producto correspondientes:

        \begin{eqnarray}
            (a + I) + (b + I) & := & (a + b) + I \\
            (a + I) \cdot (b + I) & := & (a \cdot b) + I
        \end{eqnarray}

        entonces la función:

        \begin{eqnarray*}
            \varphi & := & R \to \sfrac{R}{I} \\
             & & a \to a + I
        \end{eqnarray*}

        es un epimorfismo
        \todo{Verificación asignada a tarea.}
    \end{observacion}

    \begin{observacion}
        Tenemos que $\sfrac{R}{I}$ es un anillo con la suma y productos correspondientes:

        \begin{eqnarray}
            (a + I) + (b + I) & := & a + b + I \\
            (a + I) \cdot (b + I) & := & ab + I
        \end{eqnarray}

        La función:

        \begin{eqnarray}
            \varphi : R & \to & \sfrac{R}{I} \\
            a & \to & a + I
        \end{eqnarray}

        es epimorfismo.
    \end{observacion}

    \begin{proof}
        Primero checamos que $\varphi$ es un homomorfismo:

        \begin{eqnarray*}
            \varphi(a+b) & = & a+b+I = (a+I) + (b+I) = \varphi(a) + \varphi(b) \\
            \varphi(ab) & = & ab+I =(a+I)(b+I) = \varphi(a) \varphi(b)
        \end{eqnarray*}

        y por contrucción $\varphi$ es sobre, por lo tanto $\varphi$ es un epimorfismo.
    \end{proof}

    \todo{Tarea: Verificar que el nucleo de $\varphi$ definido como $\ker{\varphi} = \{ a \in R \mid \varphi(a) = 0 + I \}$ coincide con el ideal I.}

    \subsection{Teoremas de Isomorfismos de Anillos}

        \begin{teorema}
            Sea $\varphi: R \to R'$ un epimorfismo de anillos y $k = ker{\varphi}$, entonces $\sfrac{R}{k}$ es isomorfo con $R'$ es decir:

            \begin{equation}
                \sfrac{R}{k} \cong R'
            \end{equation}
        \end{teorema}

        \begin{teorema}
            Sea $R$ anillo, sean $A$ un subconjunto de $R$ ($A$ subanillo de $R$), y sea $B$ un ideal de $R$. Entonces $A+B$ es un subanillo de $R$ ideal de $A$, ademas:

            \begin{equation}
                \sfrac{A+B}{B} \cong \sfrac{A}{A \cap B}
            \end{equation}
        \end{teorema}

        \begin{teorema}
            Sean $I, J$ ideales del anillo $R$ con $I \subset J$, entonces $\sfrac{J}{I}$ es ideal de $\sfrac{R}{I}$, ademas:

            \begin{equation}
                \sfrac{\sfrac{R}{I}}{\sfrac{J}{I}} \cong \sfrac{R}{J}
            \end{equation}
        \end{teorema}

        \begin{ejemplo}
            Sea $n \in \mathbbm{N}$, $n > 1$ y

            \begin{eqnarray*}
                \varphi: \mathbbm{Z} & \to & \mathbbm{Z}' \\
                a & \to & [a]
            \end{eqnarray*}

            es decir, $\varphi$ es un epimorfismo con:

            \begin{eqnarray*}
                [a+b] & = & [a] + [b] \\[2pt]
                [ab] & = & [a][b]
            \end{eqnarray*}

            su nucleo es:

            \begin{eqnarray*}
                ker{\varphi} & = & \left\{ a \in \mathbbm{Z} \mid \varphi(a) = [a] = [0] \right\} \\
                & = & \left\{ a \in \mathbbm{Z} \mid a \cong 0 \mod{n} \right\} \\
                & = & \left\{ a \in \mathbbm{Z} \mid \sfrac{n}{a} \right\} \\
                & = & \left\{ a \in \mathbbm{Z} \mid a = n z, z\in \mathbbm{Z} \right\} = n \mathbbm{Z}
            \end{eqnarray*}

            por lo que aplicando el primer teorema de isomorfismos tenemos:

            \begin{equation*}
                \sfrac{\mathbbm{Z}}{n \mathbbm{Z}} \cong \mathbbm{Z}'
            \end{equation*}
        \end{ejemplo}

        \begin{ejemplo}
            Sea $F$ un campo (anillo conmutativo con división), entonces $\{0\}$ es un ideal de $F$. Sea $I$ un ideal diferente de este, $I \ne \{ 0 \}$.

            Tenemos $a \in I, a \ne 0$, entonces $a^{-1} a = 1$, con $a^{-1} \in F$, por lo que $a^{-1} a = 1 \in I$.

            Sea $r \in F$, por lo que $r = 1 \cdot r$, con $1 \in I$ y $r \in F$, pero $r$ es cualquier elemento de $F$, por lo que podemos decir que $F \subseteq I$, pero por definición $I \subseteq F$, por lo que este idel es el mismo campo.

            \begin{equation*}
                F = I
            \end{equation*}
        \end{ejemplo}

        \begin{ejemplo}
            Sea $R = \left\{ f:[0, 1] \to R \mid f es continua \right\}$. Definimos tambien:

            \begin{eqnarray*}
                (f+g)(x) & := & f(x) + g(x) \\
                (f \cdot g) & := & f(x) \cdot g(x)
            \end{eqnarray*}

            $R$ es un anillo con identidad.

            Definimos el nucleo como:

            \begin{equation*}
                I := \left\{ f \in R \mid f \left( \frac{1}{2} \right) = 0 \right\}
            \end{equation*}

            con un $\varphi$:

            \begin{eqnarray}
                \varphi: R & \to & R \\
                f & \to & f \left( \frac{1}{2} \right)
            \end{eqnarray}

            esto implica que:

            \begin{equation*}
                \sfrac{R}{I} \cong R
            \end{equation*}

            \todo{Demostración asignada a tarea}
        \end{ejemplo}
