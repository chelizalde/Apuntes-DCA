%--------------------------------------
%       CONFIGURACIONES
%--------------------------------------
\documentclass{tufte-book}
\usepackage[spanish,es-noquoting]{babel}
\usepackage[utf8]{inputenc}
\usepackage{amsmath}
\usepackage{amsthm}
\usepackage{amssymb}
\usepackage{graphicx}
\usepackage{bbm}
\usepackage{enumerate}
\usepackage{tikz}
\usepackage{todonotes}
\usepackage{xfrac}

\usetikzlibrary{shapes, arrows, decorations.markings}

\setcounter{secnumdepth}{0}
\numberwithin{equation}{chapter}

\newtheorem{definicion}{Definición}
\numberwithin{definicion}{chapter}
\newtheorem{lema}{Lema}
\numberwithin{lema}{chapter}
\newtheorem{teorema}{Teorema}
\numberwithin{teorema}{chapter}
\newtheorem{ejemplo}{Ejemplo}
\numberwithin{ejemplo}{chapter}
\newtheorem{proposicion}{Proposición}
\numberwithin{proposicion}{chapter}
\newtheorem{observacion}{Obsevación}
\numberwithin{observacion}{chapter}
\newtheorem{corolario}{Corolario}
\numberwithin{corolario}{chapter}
%--------------------------------------
%       TITULO
%--------------------------------------
\title{Matemáticas \\ Cinvestav}
\author{Generación 2014}
%--------------------------------------
%       INICIO DEL DOCUMENTO
%--------------------------------------
\begin{document}
  \maketitle
  \tableofcontents
%--------------------------------------
%       ALGEBRA ABSTRACTA
%--------------------------------------
  \part{Algebra abstracta}
    \chapter{Grupos}
      \section{Definici\`{o}n de grupo}
\paragraph{Definici\`{o}}
Un conjunto no vacio G en el que esta definida una operacion $*$ tal que va a mapear el producto cartesiano y los va amandar.\[ *:G\times G\rightarrow G  \].
\[ (a,b)\rightarrow (a*b) \]
\paragraph{Propiedad}
\begin{enumerate}
  \item $a*b \in G \forall a,b \in G$
  \item $a*b(b*c)=(a*b)*c \forall a,b,c \in G$
  \item $\exists  e \in G \therefore a*e=e*a=a\forall e\in G$ $"e"$ se le llama identidad o identidad de a
  \end{enumerate}
  \subparagraph{Ejemplo}
  \begin{enumerate}
    \item $\mathbb{Z}$ 
    \item Los racionales $ \mathbb{Q} $ con la suma
    \item $\mathbb{Q} ^{*} =  \mathbb{Q} \{0\} $ con la multiplicacion
    \item $G=\{e\} $ con la opercaion $e*e=e\in G$
    \item
    \item El conjunto de Matrices $G(n,\mathbb{R})$ es un grupo NO CONMUTATIVO 
    $A,b\in G(n,\mathbb{R} )$
    \item Son las matrices  
  \end{enumerate}

      \newpage
\section{Grupos abelinos}
  \paragraph{Definicion}
  Se dice que un grupo G es \underline{abeliano} si solo si $a*b=b*a$ 
  \subparagraph{Ejemplo}
  El conjunto $\mathbb{Z} \diagup \mathbb{Z} _{n}$ (clase de equivalencia)
 \subparagraph{Ejercicios}
 \begin{enumerate}
   \item Considere a $\mathbb{Z}$ con el producto usual Es $\mathbb{Z}$ un grupo?
   \item Considere a $\mathbb{Z}^{*}(incluye 0)$ con el producto usual es $\mathbb{Z} ^{*}$?
   \item Sea $G=\mathbb{R} \diagdown \{ 0\} $ si definimos $a\times b=a^{2}b$ G es un Grupo?
    \end{enumerate}
 \paragraph{Definiciones}
   Orden de un grupo es el numero de elementos que tiene dicho Grupo y se denota $|G| $ 
  Un Grupo G sera finito si tiene elementos finitos de elementos sea infinito
  \subparagraph{Ejemplos}
\begin{description} 
  \item[Proposicion] Si G es un grupo entonces
  \begin{enumerate}
    \item El elemento identidad es uinico
    \item $\forall a\in G a^{-1}$ es unico
    \item $\forall a,b \in G(ab)^{-1}=b^{-1}a^{-1}$
    \item En general $(a_{1}\cdot a_{2}\cdot  \dots a_{n}) ^{-1} = (a_{n}^{-1}\cdot a_{n-1}^{-1}\cdot \dots a_{2}^{-1}\cdot a_{1}^{-1} ) \forall  a\in G $ 
  \end{enumerate}
  \item[Proposicion] Sea G un grupo $\forall a,b,c\in G$
  \begin{enumerate}
    \item $ab=ac \Rightarrow b=c$
    \item $ba=ca \Rightarrow b=c$
  \end{enumerate}
\end{description}
\subparagraph{Verificacion}
\begin{enumerate}
  \item $b=eb=(aa^{-1})b=a^{-1}(ab)=a^{-1}(ac)=(a^{-1}a)c=ec=c$
  \item $b=be=b(aa^{-1})=(ba)a^{-1}=(ca)a^{-1}=c(aa^{-1})=ce=c$
\end{enumerate}

      \newpage
\section{Subgrupo}
\paragraph{Definici\`on}
  Conjunto no vac\`io H de un grupo G, se llama Subgrupo si H mismo forma un grupo respecto a la operaci\`{o} de G.
  Cuando H es subgrupo de G se denota $H< G$  \`{o} $G> H$.
  \paragraph{Observaci\`on}
  Todo grupo tiene aut\`omaticamente dos subgrupos tribiales  $G \& \{ e \} $
    \paragraph{Propoci\`on}
  Un subconjunto no vaio $H\subset G$ es un subgrupo de G ssi H es cerrado respecto a la operaci\`{o}n G \& $a\in H\Rightarrow a^{-1}\in a^{-1}\in H$ 
  \\ $\Rightarrow$ \subparagraph{Necesidad} Como H es un subgrupo de G, H es un grupo y tiene inversa
  \\  $\Leftarrow$ \subparagraph{Suficiencia} H es cerrado, no vacio \& y el  inverso esta en $H\forall a\in H=aa^{-1}(H es cerrado)\Rightarrow aa^{-1}=e\in H$ 
  \\
  \\ Ademas para $a,b,c\in H$  $a(bc)=(ab)c$ $H\in G$

  \paragraph{Ejercicio} Sea $G=\mathbb{Z} $ con la  suma usual \& sea H el conjunto de enteros pares.
  \\ \[H=\{2n\diagup n\in \mathbb{Z} \}   \] H es un subgrupo?
  \paragraph{Sean a,b$\in$H} 
  $a=2q$ , $q\in \mathbb{Z} $  $b=2\acute{q}$ $\acute{q}\in \mathbb{Z}  $
  \\
  \\ $a+b=2q+2q=2(q+q')=2q'' $
  

      \newpage
\section{Subgrupo Normal}
	\begin{definicion}
		Un grupo $N$ de $G$ se dice que es un Subgrupo Normal de $G$ denotado por:

		\begin{equation}
			N \triangle G \nonumber
		\end{equation}
		
		si $\forall g \in G$ y $\forall n \in N$, se tiene que:

		\begin{equation}
			g n g^{-1} \in N
		\end{equation}
	\end{definicion}

	\begin{lema}
		$N$ es un subgrupo de G, si y solo si:
		\begin{equation}
			g N g^{-1} = N \quad \forall g \in G
		\end{equation}
	\end{lema}

	\begin{proof} 
		\begin{enumerate}[i]
			\item Si $g N g^{-1} = N \quad \forall g \in G$, entonces en particular:
			
			\begin{equation}
				g N g^{-1} \subseteq N \nonumber
			\end{equation}

			por lo que $g N g^{-1} \in N \quad \forall n \in N$, por lo tanto:

			\begin{equation}
				N \triangle G \nonumber
			\end{equation}
			
			\item Si $N$ es un subgrupo normal de $G$, entonces:

			\begin{equation}
				g N g^{-1} \in N \nonumber
			\end{equation}

			Si $g \in G \quad \forall n \in N$, entonces $g N g^{-1} \subseteq N$.

			Por otro lado $g^{-1} N g = g^{-1} N (g^{-1})^{-1} \subseteq N$, ademas:

			\begin{equation}
				N = e N e = g \left( g^{-1} N g \right) g^{-1} = g N g^{-1} \nonumber
			\end{equation}

			por lo tanto:

			\begin{equation}
				g N g^{-1} = N \nonumber
			\end{equation}
		\end{enumerate}
	\end{proof}
    \chapter{Homorfismos de grupo}
      \section{Definicion de Homomorfismos } 
	\begin{definicion}
	Una aplicaion $\varepsilon :G\rightarrow \bar{G} $ $G$ es un grupo con operacion ($\bullet$ ) \& $\bar{G} $ es un grupo con operacion ($\blacksquare $) \\

	Se dice que es un homomorfismo si para $a,b\in  G$ cuales quiera se tiene que:
	  \begin{equation}
	    \varepsilon (a\bullet b)=\varepsilon (a)\blacksquare \varepsilon (b) 
	  \end{equation}
	\end{definicion} 
	
	\begin{ejemplo}
	$G=\mathbb{R} ^{+}\diagdown \{ 0\} $ 
	bajo la multiplicacion \& sea
	 $\bar{G}=\mathbb{R}   $ 
	 bajo la adicion definimos $\varepsilon :G \rightarrow \bar{G} $
	  como
	   $ \varepsilon :\mathbb{R} ^{+\diagdown \{ 0\} } \Rightarrow \mathbb{R} \rightarrow ln(r)$ 
	   sean 
	   $r_{1}r_{2}\in\mathbb{R} ^{+\diagdown \{ 0\} }$ 
	   t.q.
	    $\varepsilon (r_{1}\bullet r_{2})=ln(r_{1}r_{2})=ln(r_{1})+ln(r_{2})=\varepsilon (r_{1})+\varepsilon (r_{1})$ 
	    por tanto $\varepsilon $ es un homomorfismo.
	    \end{ejemplo}
	  \begin{lema}
	  Supongamos que $G$ es un grupo \&  que $N$ es un subgrupo de $G$ \\
	  Definamos la siguiente aplicaci\`on 
	  \begin{equation}
	    \varepsilon :G\rightarrow G\diagdown N \nonumber
	  \end{equation}
	  entonces $\varepsilon $ es 
	  \begin{equation}
	    x\rightarrow N_{x}
	  \end{equation}
	  un Homomorfismo
	  \end{lema}
	  \begin{definicion}
	  Un homomorfismo
      $\varepsilon :G\rightarrow \acute{G} $     
	  se dice que :
	  \begin{enumerate}[a]
	    \item Monorfismo si es 1-1 (inyectiva)
	    \item Epimorfismo si es suprayectiva
	    \item Isomorfismo si es biyectiva
	  \end{enumerate}
	  \end{definicion}
	  \begin{definicion}
	  Si $\varepsilon :G\rightarrow \acute{G} $ es un isomorfismo, decimos que  $G$ \& $\acute{G} $ son isomorfos \& escribimos 
	  \begin{equation}
	    G\cong \acute{G}  
	  \end{equation}
	  \end{definicion}
	\begin{proposicion}
	Si $\varepsilon :G\rightarrow \acute{G} $ es un homomorfismo entonces 
	\begin{equation}
	  Im\varepsilon < \acute{G} 
	\end{equation}
	dondela
	\begin{equation}
    Im\varepsilon =\{ y\in \acute{G}| \varepsilon _{(x)}=y,x\in G\} \subset G 	
	\end{equation}
	\end{proposicion}
	\begin{proof}
    \begin{enumerate}[i]
      \item Sean 
      \begin{equation}
        y_{1},y_{2}\in Im\varepsilon \Rightarrow y_{1}=\varepsilon (x_{1}) \& y_{2}=\varepsilon (x_{2})\in \acute{G} \nonumber
      \end{equation}
      \begin{equation}
      x_{1},x_{2}\in G \nonumber
      \end{equation}
      \begin{equation}
         y_{1},y_{2}=\varepsilon x_{1}\varepsilon x_{2}=\varepsilon (x_{1}x_{2} \Rightarrow y_{1},y_{2}\in Im\varepsilon  \nonumber
      \end{equation}
    \end{enumerate}
	\end{proof}
	

    \chapter{Anillos}
      \newpage
\section{Definiciones}
    \begin{definicion}
        Un conjunto no vacio $R$ es un anillo si tiene definidas dos operaciones $(+, \cdot)$ tales que:
        \begin{enumerate}[a)]
            \item $a, b \in R \implies a+b \in R$
            \item $a + (b + c) = (a + b) + c \quad \forall a, b, c \in R$
            \item $a + b = b + a \quad \forall a, b, \in R$
            \item $\exists 0 \in R \text{ tal que } a + 0 = a \quad \forall a \in R$
            \item $\exists b \in R \text{ tal que } a + b = 0 \quad \forall a \in R$
        \end{enumerate}

        De estas propiedades podemos concluir que $R$ es un grupo abeliano con respecto a $(+)$, pero aun tenemo lo siguiente:

        \begin{enumerate}[1)]
            \item $a, b \in R \implies a \cdot b \in R$
            \item $a \cdot (b \cdot c) = (a \cdot b) \cdot c \quad \forall a, b, c\in R$
        \end{enumerate}

        De estas propiedades podemos concluir que $R$ es un semigrupo con respecto a $(\cdot)$, y ademas:

        \begin{enumerate}[i)]
            \item $a \cdot (b + c) = a \cdot b + a \cdot c \quad \forall a, b, c \in R$
            \item $(b + c) \cdot a = b \cdot a + c \cdot a \quad \forall a, b, c \in R$
        \end{enumerate}
    \end{definicion}

    \begin{definicion}
        Diremos que un anillo $R$ es un anillo con identidad si existe un $1 \in R$, con $1 \ne 0$, tal que:

        \begin{equation}
            a \cdot 1 = 1 \cdot a \quad \forall a \in R
        \end{equation}
    \end{definicion}

    \begin{definicion}
        Un anillo $R$ es un anillo conmutativo si:

        \begin{equation}
            a \cdot b = b \cdot a \quad \forall a, b \in R
        \end{equation}
    \end{definicion}

    \begin{definicion}
        Sea $R$ un anillo y $a \in R$ con $a \ne 0$, diremos que $a$ es divisor de cero, si existe $b \in R$ con $b \ne 0$ tal que:

        \begin{equation}
            a \cdot b = 0 \quad \text{(divisor por la derecha)}
        \end{equation}

        o bien si existe un $c \in R$ con $c \ne 0$, tal que:

        \begin{equation}
            c \cdot a = 0 \quad \text{(divisor por la izquierda)}
        \end{equation}
    \end{definicion}

    \begin{definicion}
        Sea $R$ un anillo con identidad. Diremos que $R$ es un anillo con división si existe $b \in R$ tal que:

        \begin{equation}
            a \cdot b = b \cdot a = 1 \quad \forall 0 \ne a \in R
        \end{equation}
    \end{definicion}

    \begin{definicion}
        Un campo es un anillo con división, que ademas es conmutativo, es decir, un campo es un grupo abeliano con respecto a $(+)$ y a $(\cdot)$.
    \end{definicion}

    \begin{definicion}
        Un anillo conmutativo con identidad es un dominio entero si:

        \begin{equation}
            a \cdot b = 0 \implies a = 0 o b = 0
        \end{equation}

        Esto quiere decir que no existen divisores de cero.
    \end{definicion}

    \begin{corolario}
        Si $p$ es primo, entonces $\mathbbm{Z}_p$ es campo.
    \end{corolario}

    \todo{Tarea:
    Sea $\mathcal{M}_{2 \times 2}(\mathbbm{R}) = \left\{ \left. \begin{pmatrix} a & b \\ c & d \end{pmatrix} \right| a, b, c, d \in \mathbbm{R} \right\}$
    Verificar que $\mathcal{M}_{2 \times 2}$ es un anillo con identidad no conmutativo y ademas no es dominio entero.}

    \begin{proposicion}
        Sea $R$ un anillo y sean $a, b \in R$, entonces:

        \begin{enumerate}[a)]
            \item $a \cdot 0 = 0 \cdot a = 0$
            \item $a \cdot (-b) = (-a) \cdot b = - (a \cdot b)$
            \item $(-a)(-b) = (-(-a)) \cdot b = a \cdot b$
            \item $1 \in R \implies (-1) \cdot a = -a$
        \end{enumerate}
    \end{proposicion}

    \begin{proof}
        \begin{equation*}
            a \cdot 0 + 0 = a \cdot 0 = a \cdot(0 + 0) = a \cdot 0 + a \cdot 0 \implies 0 = a \cdot 0
        \end{equation*}

        \begin{equation*}
            a \cdot (-b) + a \cdot b = a \cdot (-b + b) = a \cdot (0) = 0 \implies a \cdot (-b) = -(a \cdot b)
        \end{equation*}

        \begin{multline*}
            (-a) \cdot (-b) + (-a) \cdot (b) = (-a) \cdot (-b + b) = (-a) \cdot (0) = 0 \\
            \implies (-a) \cdot (-b) = -(-a) \cdot (b) = (-(-a)) \cdot (b)
        \end{multline*}
    \end{proof}

      
\newpage
\section{Ideales, Homomorfismos y Anillos}
    \begin{definicion}
        Una función $\varphi: R \to R'$ es un homomorfismo si:

        \begin{enumerate}
            \item $\varphi(a) + \varphi(b) = \varphi(a + b)$
            \item $\varphi(a) \cdot \varphi(b) = \varphi(a \cdot b)$
        \end{enumerate}
    \end{definicion}

    \begin{definicion}
        Sea $\varphi: R \to R'$ homomorfismo de anillos, entonces:
        \begin{enumerate}
            \item $\varphi$ es monomorfismo si es inyectivo
            \item $\varphi$ es epimorfismo si es suprayectivo
            \item $\varphi$ es isomorfismo si es biyectivo
        \end{enumerate}
    \end{definicion}

    \begin{definicion}
        El nucleo de $\varphi$ es $\ker{\varphi} = \left\{ x \in R \mid \varphi(x) = 0 \right\}$
    \end{definicion}

    \begin{proposicion}
        Sea $\varphi:R \to R'$, es un homomorfismo de anillos, entonces:

        \begin{enumerate}
            \item $\ker{\varphi}$ es un subgrupo aditivo
            \item $rk, kr \in \ker{\varphi} \quad \forall k \in \ker{\varphi} \quad r \in R$
        \end{enumerate}
    \end{proposicion}

    \begin{proof}
        Sea $k \in \ker{\varphi}$ y $r \in R$

        \begin{eqnarray*}
            \varphi(rk) & = & \varphi(r) \cdot \varphi(k) \\
            & = & \varphi(r) \cdot 0 = 0
        \end{eqnarray*}

        \begin{eqnarray*}
            \varphi(kr) & = & \varphi(k) \cdot \varphi(r) \\
            & = & 0 \cdot \varphi(r) = 0
        \end{eqnarray*}

        \begin{equation*}
            \therefore kr, rk \in \ker{\varphi}
        \end{equation*}
    \end{proof}

    \begin{definicion}
        Sea $R$ un anillo e $I$ un subconjunto de $R$, se dice que es un ideal de $R$ si:

        \begin{enumerate}[i)]
            \item $I$ es un subgrupo aditivo de $R$
            \item Dados $r \in R$ y $a \in I$, tenemos que $ra \in I$ y $ar \in I$
        \end{enumerate}

        a esto se le conoce como propiedad de absorción.
    \end{definicion}

    \begin{corolario}
        Si $\varphi: R \to R'$ es un homomorfismo, entonces $k = \ker{\varphi}$ es un ideal de $R$.
    \end{corolario}

    \begin{definicion}
        Sea $R$ anillo e $I$ un ideal de $R$, entonces $\sfrac{R}{I}$ (anillo cociente) es un grupo con la suma de clases de equivalencia.

        \begin{equation}
            (a + I) + (b + I) = (a + b) + I \quad \forall a, b \in R
        \end{equation}
    \end{definicion}

    \begin{definicion}
        Definimos el producto como:

        \begin{equation}
            (a + I)(b + I) = ab + I \quad \forall a, b \in R
        \end{equation}
    \end{definicion}

    \begin{observacion}
        Sea $R$ un anillo, tenemos que $\{ 0 \}$ y $R$ son ideales de $R$ y triviales.
    \end{observacion}

    \begin{definicion}
        Si $I$ es un ideal de $R$ e $I \ne R$, decimos que $I$ es un ideal propio.
    \end{definicion}

    \begin{observacion}
        Sea $R$ un anillo con identidad e $I$ ideal de $R$ tal que $1 \in I$, entonces $I = R$.
    \end{observacion}

    \begin{proof}
        Por definicion $I \subseteq R$, por lo que solo resta demostrar que $R \subset I$.

        Sea $a \in R$, entonces:

        \begin{equation*}
            a = a \cdot I \quad 1 \in I \quad a \in R
        \end{equation*}

        por lo tanto $a \in I$, $\therefore R \subseteq I$.
    \end{proof}

    \begin{definicion}
        Sea $R$ anillo conmutativo con identidad. Un ideal principal, es un ideal de la forma $(a)$, para algun $a \in R$.
    \end{definicion}

    \begin{definicion}
        Sea $R$ un dominio enterio, diremos que $R$ es un dominio de ideales principales, si todos los ideales de $R$ son principales.
    \end{definicion}

    \begin{observacion}
        Sean $I, J$ ideales del anillo $R$, si definimos:

        \begin{equation}
            I + J = \left\{ a + b \mid a \in I, b \in J \right\}
        \end{equation}

        Verificar que:

        \begin{enumerate}
            \item $I + J$ es un ideal
            \item $I \cap J$ es un ideal
            \item $IJ := \left\{ \sum_{i=1}^n a_i b_i \mid n \in \mathbbm{N}, a \in I, b \in J \right\}$ es un ideal.
        \end{enumerate}

        \todo{Verificación asignada a tarea.}
    \end{observacion}

    \begin{observacion}
        Tenemos que $\sfrac{R}{I}$ es un anillo con la suma y el producto correspondientes:

        \begin{eqnarray}
            (a + I) + (b + I) & := & (a + b) + I \\
            (a + I) \cdot (b + I) & := & (a \cdot b) + I
        \end{eqnarray}

        entonces la función:

        \begin{eqnarray*}
            \varphi & := & R \to \sfrac{R}{I} \\
             & & a \to a + I
        \end{eqnarray*}

        es un epimorfismo
        \todo{Verificación asignada a tarea.}
    \end{observacion}

    \begin{observacion}
        Tenemos que $\sfrac{R}{I}$ es un anillo con la suma y productos correspondientes:

        \begin{eqnarray}
            (a + I) + (b + I) & := & a + b + I \\
            (a + I) \cdot (b + I) & := & ab + I
        \end{eqnarray}

        La función:

        \begin{eqnarray}
            \varphi : R & \to & \sfrac{R}{I} \\
            a & \to & a + I
        \end{eqnarray}

        es epimorfismo.
    \end{observacion}

    \begin{proof}
        Primero checamos que $\varphi$ es un homomorfismo:

        \begin{eqnarray*}
            \varphi(a+b) & = & a+b+I = (a+I) + (b+I) = \varphi(a) + \varphi(b) \\
            \varphi(ab) & = & ab+I =(a+I)(b+I) = \varphi(a) \varphi(b)
        \end{eqnarray*}

        y por contrucción $\varphi$ es sobre, por lo tanto $\varphi$ es un epimorfismo.
    \end{proof}

    \todo{Tarea: Verificar que el nucleo de $\varphi$ definido como $\ker{\varphi} = \{ a \in R \mid \varphi(a) = 0 + I \}$ coincide con el ideal I.}

    \subsection{Teoremas de Isomorfismos de Anillos}

        \begin{teorema}
            Sea $\varphi: R \to R'$ un epimorfismo de anillos y $k = ker{\varphi}$, entonces $\sfrac{R}{k}$ es isomorfo con $R'$ es decir:

            \begin{equation}
                \sfrac{R}{k} \cong R'
            \end{equation}
        \end{teorema}

        \begin{teorema}
            Sea $R$ anillo, sean $A$ un subconjunto de $R$ ($A$ subanillo de $R$), y sea $B$ un ideal de $R$. Entonces $A+B$ es un subanillo de $R$ ideal de $A$, ademas:

            \begin{equation}
                \sfrac{A+B}{B} \cong \sfrac{A}{A \cap B}
            \end{equation}
        \end{teorema}

        \begin{teorema}
            Sean $I, J$ ideales del anillo $R$ con $I \subset J$, entonces $\sfrac{J}{I}$ es ideal de $\sfrac{R}{I}$, ademas:

            \begin{equation}
                \sfrac{\sfrac{R}{I}}{\sfrac{J}{I}} \cong \sfrac{R}{J}
            \end{equation}
        \end{teorema}

        \begin{ejemplo}
            Sea $n \in \mathbbm{N}$, $n > 1$ y

            \begin{eqnarray*}
                \varphi: \mathbbm{Z} & \to & \mathbbm{Z}' \\
                a & \to & [a]
            \end{eqnarray*}

            es decir, $\varphi$ es un epimorfismo con:

            \begin{eqnarray*}
                [a+b] & = & [a] + [b] \\[2pt]
                [ab] & = & [a][b]
            \end{eqnarray*}

            su nucleo es:

            \begin{eqnarray*}
                ker{\varphi} & = & \left\{ a \in \mathbbm{Z} \mid \varphi(a) = [a] = [0] \right\} \\
                & = & \left\{ a \in \mathbbm{Z} \mid a \cong 0 \mod{n} \right\} \\
                & = & \left\{ a \in \mathbbm{Z} \mid \sfrac{n}{a} \right\} \\
                & = & \left\{ a \in \mathbbm{Z} \mid a = n z, z\in \mathbbm{Z} \right\} = n \mathbbm{Z}
            \end{eqnarray*}

            por lo que aplicando el primer teorema de isomorfismos tenemos:

            \begin{equation*}
                \sfrac{\mathbbm{Z}}{n \mathbbm{Z}} \cong \mathbbm{Z}'
            \end{equation*}
        \end{ejemplo}

        \begin{ejemplo}
            Sea $F$ un campo (anillo conmutativo con división), entonces $\{0\}$ es un ideal de $F$. Sea $I$ un ideal diferente de este, $I \ne \{ 0 \}$.

            Tenemos $a \in I, a \ne 0$, entonces $a^{-1} a = 1$, con $a^{-1} \in F$, por lo que $a^{-1} a = 1 \in I$.

            Sea $r \in F$, por lo que $r = 1 \cdot r$, con $1 \in I$ y $r \in F$, pero $r$ es cualquier elemento de $F$, por lo que podemos decir que $F \subseteq I$, pero por definición $I \subseteq F$, por lo que este idel es el mismo campo.

            \begin{equation*}
                F = I
            \end{equation*}
        \end{ejemplo}

        \begin{ejemplo}
            Sea $R = \left\{ f:[0, 1] \to R \mid f es continua \right\}$. Definimos tambien:

            \begin{eqnarray*}
                (f+g)(x) & := & f(x) + g(x) \\
                (f \cdot g) & := & f(x) \cdot g(x)
            \end{eqnarray*}

            $R$ es un anillo con identidad.

            Definimos el nucleo como:

            \begin{equation*}
                I := \left\{ f \in R \mid f \left( \frac{1}{2} \right) = 0 \right\}
            \end{equation*}

            con un $\varphi$:

            \begin{eqnarray}
                \varphi: R & \to & R \\
                f & \to & f \left( \frac{1}{2} \right)
            \end{eqnarray}

            esto implica que:

            \begin{equation*}
                \sfrac{R}{I} \cong R
            \end{equation*}

            \todo{Demostración asignada a tarea}
        \end{ejemplo}

    \chapter{Dominios}
%--------------------------------------
%       ALGEBRA LINEAL
%--------------------------------------
  \part{Algebra Lineal}
    \chapter{Espacios vectoriales}
    \newpage 
\section{Espacios Vectoriales}

	\begin{definicion}

Un conjunto no vaci\`o $V$  se dice que es un espacio vectorial sobre un campo $F$ si $V$ es un grupo abeliano respecto a una operacai\`on que se denota por $+$, y si para todo $\alpha \in F$, $v\in V$ est\`a definido un elemento, escrito como $\alpha v$, de $V$ con las siguiente propiedades: 

\begin{enumerate}
  \item $\alpha (v+w)=\alpha v+\alpha w$
  \item $(\alpha+\beta )v=\alpha v+\beta v$
  \item $\alpha(\beta v)=(\alpha\beta)v$
  \item $1v=v$
\end{enumerate}

Una opercai\`on denominada multiplicaci\`on por escalares que asocia a cada escalar $c\in F  $, un vector $c \alpha \in V$ de manera que:

\begin{enumerate}[a]
	\item $(c1c2)\alpha=c1(c2\alpha)$
	\item $c(\alpha+\beta)=c\alpha+c\beta \forall c\inF \alpha,\beta\inV$
	\item $(c_1+c_2) \alpha = c_1  \alpha + c_2 \alpha \forall c_1 ,c_2 \inF \alpha \in V$
	\item $1\alpha = \alpha \forall \alpha \in V$	
\end{enumerate}
		
		\begin{definicon}
		Sea F un campo & sea $n \in N$
		
		\begin{equation}
		F^n =\{ (x_1 , \dots \dots , x_n) \ | x_1 \in F }
		\end{equation}
    \newpage
\section{Subespacio}
\begin{definicion}	
Un subespacio de un espacio vectorial $V$ sobre $F$ es un subconjunto $W$ de $V$ que con las operaciones heredadas de $W$ es el mismo espacio vectorial sobre $F$.
		
	\end{definicion}	
\emph{Nota : si $V$ es un e.v. , $V=\{ \vec{0} \}$ se denomina subespacio triviales de $V$} 

\begin{proposicion}
 	Un subconjuto no vacio $W$ de $V$ es un subespacio vectorial ssi $W$ es cerrado con respectoa los operaciones de $V$
 \end{proposicion} 

 $\rightarrow$ Si $W$ es s.e.v. de $V$ por defenici\`'on es e.v. y sus operaciones son cerradas.
 $\leftarrow$ 

 
    \newpage
\section{Combinacion lineal}
\begin{definicion}
	Se dice que $\beta \in V $ es una combinaci\`on lineal de vetores $\alpha , \dots , \alpha_n$ exiten $c_1, \dots , c_n \in F$ tales que
			\begin{equation}
				\beta = \sum\limits{i=1}{n} ci \ di \nonumber
			\end{equation}

	\end{definicion}

	

 
 
 \begin{definicion}
 	Sea $\alpha_1 \dots \alpha_k$ en $V$ \& $\mathcal{L}(\alpha_1 \dots \alpha_k)=\{ \beta	| \beta$ es combinaci\`on lineal de $\alpha_1 \ \dots \ \alpha_k$ esto es un s.e.v. de $V$ \& se llama subespacio generado por $\alpha_i \leq i = k$ o bien se dice que $\alpha_1 \dots \alpha_k$ genera a $\mathcal{L}$
 \end{definicion}

En general \\ Si $A \neq 0$	\& si $ A \subset V $ entonces $\mathcal{L} (A)= \{ \beta | \beta $ es combinaci\`on lineal de los elemntos de $A \ \}$
\begin{proposicion}
			La intersecion de cualquier coleci\`on de subespacio de $V$ es un subespacio de $V$	 	 										
	 	 									\end{proposicion}	 	 									
\begin{proof}
Sea $ \{ Wa | \alpha \in I \} $ \& sea $ W \triangleq \bigcap W \alpha $
	
		\begin{enumerate}[i]
			\item  $w$ no es vacio  $\in W$ pues $\vec{a} \in W \alpha \ \forall \alpha \in I$

			\item $\alpha , \beta \in W \rightarrow \alpha , \beta \in W \alpha \ \forall \ \alpha \in I$  $\rightarrow \ \alpha + \beta \ \in W \alpha \ \forall \ \alpha \in I \ \rightarrow \ \alpha + \beta \in W$

			\item  Sea $\beta \in W $ \& sea $r \in \mathcal{F} \rightarrow \ r ,\beta \in W \alpha \ \forall \alpha\in I \ \rightarrow \ r, \beta \in W$
		\end{enumerate}

\end{proof}

\begin{observacion}
 		La uni\`on de los subespacios no necesariamente es espacio vectorial

\end{observacion}

\begin{ejemplo}
	 En $\mathbb{R}^2 $ definimos \\ $W_1 = \{ ( x_1 , 0 ) | x_1 \in R \} \subset \mathbb{R}^2$ \\ $W_2 = \{ ( 0,x_2 ) | x_2 \in R \} \subset \mathbb{R}^2 $ \\ $(x_1,0)+(0,x_2)= (x_1 , x_2) \notin W_1 \cup W_2$ \\ $W_1 \cup W_2 = \{ ( x_1 , 0 ),( 0,x_2 ) \} | x_1 , x_2 \in \mathbb{R}$
\end{ejemplo}

    \newpage
\section{Suma directa}
\begin{definicion}
	Sean $S,T$ subespacios de $V$ definimos la suma de $S$ \& $I$ como 
	\begin{equation}
		S + I = \{ s + t | s \in S , t \in T \} \nonumber 
	\end{equation}
\end{definicion}

\begin{proposicion}
	 Si $S$ \& $T$ son subespacios entonces $S+T$ es un subespacio vectorial de $V$
\end{proposicion}

Tarea pendiente 
\begin{definicion}
	 Si $S$ \& $T$ son subespacios $V$ de $V$ tales que $S+T=V$ \& $S \cap T = \{ 0 \} $ decimos que $V$ es la suma directa  
 \end{definicion} 

 \begin{proof}
 	Sea $s,t$ , $ s^\prime , t^\prime $ que estan en $S$ \& $T$ respectivamente \& sea $\alpha \in V$   
 \end{proof}
    \chapter{Operadores lineales}
    \chapter{Funciones lineales}
    \chapter{Espacios duales}
    \chapter{Teorema de Caley-Hamilton}
    \chapter{Diagonalizacion}
    \chapter{Forma canónica de Jordan}
    \chapter{Vectores propios generalizados}
%--------------------------------------
%       ECUACIONES DIFERENCIALES
%--------------------------------------
  \part{Ecuaciones difereciales}
    \chapter{Resolución de ecuaciones diferenciales}
    \chapter{Existencia y unicidad de la solución de una ED}
    \chapter{Solución aproximada}
    \chapter{Relación entre soluciones aproximadas y exactas}
%--------------------------------------
%       FIN DEL DOCUMENTO
%--------------------------------------
\end{document}
