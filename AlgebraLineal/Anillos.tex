\newpage
\section{Definiciones}
    \begin{definicion}
        Un conjunto no vacio $R$ es un anillo si tiene definidas dos operaciones $(+, \cdot)$ tales que:
        \begin{enumerate}[a)]
            \item $a, b \in R \implies a+b \in R$
            \item $a + (b + c) = (a + b) + c \quad \forall a, b, c \in R$
            \item $a + b = b + a \quad \forall a, b, \in R$
            \item $\exists 0 \in R \text{ tal que } a + 0 = a \quad \forall a \in R$
            \item $\exists b \in R \text{ tal que } a + b = 0 \quad \forall a \in R$
        \end{enumerate}

        De estas propiedades podemos concluir que $R$ es un grupo abeliano con respecto a $(+)$, pero aun tenemo lo siguiente:

        \begin{enumerate}[1)]
            \item $a, b \in R \implies a \cdot b \in R$
            \item $a \cdot (b \cdot c) = (a \cdot b) \cdot c \quad \forall a, b, c\in R$
        \end{enumerate}

        De estas propiedades podemos concluir que $R$ es un semigrupo con respecto a $(\cdot)$, y ademas:

        \begin{enumerate}[i)]
            \item $a \cdot (b + c) = a \cdot b + a \cdot c \quad \forall a, b, c \in R$
            \item $(b + c) \cdot a = b \cdot a + c \cdot a \quad \forall a, b, c \in R$
        \end{enumerate}
    \end{definicion}

    \begin{definicion}
        Diremos que un anillo $R$ es un anillo con identidad si existe un $1 \in R$, con $1 \ne 0$, tal que:

        \begin{equation}
            a \cdot 1 = 1 \cdot a \quad \forall a \in R
        \end{equation}
    \end{definicion}

    \begin{definicion}
        Un anillo $R$ es un anillo conmutativo si:

        \begin{equation}
            a \cdot b = b \cdot a \quad \forall a, b \in R
        \end{equation}
    \end{definicion}

    \begin{definicion}
        Sea $R$ un anillo y $a \in R$ con $a \ne 0$, diremos que $a$ es divisor de cero, si existe $b \in R$ con $b \ne 0$ tal que:

        \begin{equation}
            a \cdot b = 0 \quad \text{(divisor por la derecha)}
        \end{equation}

        o bien si existe un $c \in R$ con $c \ne 0$, tal que:

        \begin{equation}
            c \cdot a = 0 \quad \text{(divisor por la izquierda)}
        \end{equation}
    \end{definicion}

    \begin{definicion}
        Sea $R$ un anillo con identidad. Diremos que $R$ es un anillo con división si existe $b \in R$ tal que:

        \begin{equation}
            a \cdot b = b \cdot a = 1 \quad \forall 0 \ne a \in R
        \end{equation}
    \end{definicion}

    \begin{definicion}
        Un campo es un anillo con división, que ademas es conmutativo, es decir, un campo es un grupo abeliano con respecto a $(+)$ y a $(\cdot)$.
    \end{definicion}

    \begin{definicion}
        Un anillo conmutativo con identidad es un dominio entero si:

        \begin{equation}
            a \cdot b = 0 \implies a = 0 o b = 0
        \end{equation}

        Esto quiere decir que no existen divisores de cero.
    \end{definicion}

    \begin{corolario}
        Si $p$ es primo, entonces $\mathbbm{Z}_p$ es campo.
    \end{corolario}

    \todo{Tarea:
    Sea $\mathcal{M}_{2 \times 2}(\mathbbm{R}) = \left\{ \left. \begin{pmatrix} a & b \\ c & d \end{pmatrix} \right| a, b, c, d \in \mathbbm{R} \right\}$
    Verificar que $\mathcal{M}_{2 \times 2}$ es un anillo con identidad no conmutativo y ademas no es dominio entero.}

    \begin{proposicion}
        Sea $R$ un anillo y sean $a, b \in R$, entonces:

        \begin{enumerate}[a)]
            \item $a \cdot 0 = 0 \cdot a = 0$
            \item $a \cdot (-b) = (-a) \cdot b = - (a \cdot b)$
            \item $(-a)(-b) = (-(-a)) \cdot b = a \cdot b$
            \item $1 \in R \implies (-1) \cdot a = -a$
        \end{enumerate}
    \end{proposicion}

    \begin{proof}
        \begin{equation*}
            a \cdot 0 + 0 = a \cdot 0 = a \cdot(0 + 0) = a \cdot 0 + a \cdot 0 \implies 0 = a \cdot 0
        \end{equation*}

        \begin{equation*}
            a \cdot (-b) + a \cdot b = a \cdot (-b + b) = a \cdot (0) = 0 \implies a \cdot (-b) = -(a \cdot b)
        \end{equation*}

        \begin{multline*}
            (-a) \cdot (-b) + (-a) \cdot (b) = (-a) \cdot (-b + b) = (-a) \cdot (0) = 0 \\
            \implies (-a) \cdot (-b) = -(-a) \cdot (b) = (-(-a)) \cdot (b)
        \end{multline*}
    \end{proof}
