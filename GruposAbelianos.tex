\newpage
\section{Grupos Abelianos}
  \paragraph{Definicion}
  Se dice que un grupo G es \underline{Abeliano} si solo si $a*b=b*a$
  \paragraph{Ejemplo}
    El conjunto $\mathbb{Z} \diagup \mathbb{Z} _{n}$ (clase de equivalencia)
  \paragraph{Ejercicios}
    \begin{enumerate}
      \item Considere a $\mathbb{Z}$ con el producto usual Es $\mathbb{Z}$ un grupo?
      \item Considere a $\mathbb{Z}^{*}(incluye 0)$ con el producto usual es $\mathbb{Z} ^{*}$?
      \item Sea $G=\mathbb{R} \diagdown \{ 0\} $ si definimos $a\times b=a^{2}b$ G es un Grupo?
    \end{enumerate}
 \paragraph{Definiciones}
   Orden de un grupo es el numero de elementos que tiene dicho Grupo y se denota $|G| $ 
  Un Grupo G sera finito si tiene elementos finitos de elementos sea infinito
  \paragraph{Ejemplos}
\begin{description} 
  \item[Proposicion] Si G es un grupo entonces
  \begin{enumerate}
    \item El elemento identidad es uinico
    \item $\forall a\in G a^{-1}$ es unico
    \item $\forall a,b \in G(ab)^{-1}=b^{-1}a^{-1}$
    \item En general $(a_{1}\cdot a_{2}\cdot  \dots a_{n}) ^{-1} = (a_{n}^{-1}\cdot a_{n-1}^{-1}\cdot \dots a_{2}^{-1}\cdot a_{1}^{-1} ) \forall  a\in G $ 
  \end{enumerate}
  \item[Proposicion] Sea G un grupo $\forall a,b,c\in G$
  \begin{enumerate}
    \item $ab=ac \Rightarrow b=c$
    \item $ba=ca \Rightarrow b=c$
  \end{enumerate}
\end{description}
\paragraph{Verificacion}
\begin{enumerate}
  \item $b=eb=(aa^{-1})b=a^{-1}(ab)=a^{-1}(ac)=(a^{-1}a)c=ec=c$
  \item $b=be=b(aa^{-1})=(ba)a^{-1}=(ca)a^{-1}=c(aa^{-1})=ce=c$
\end{enumerate}
