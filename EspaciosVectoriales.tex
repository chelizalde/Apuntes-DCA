\newpage 
\section{Espacios Vectoriales}

	\begin{definicion}

Un conjunto no vaci\`o $V$  se dice que es un espacio vectorial sobre un campo $F$ si $V$ es un grupo abeliano respecto a una operacai\`on que se denota por $+$, y si para todo $\alpha \in F$, $v\in V$ est\`a definido un elemento, escrito como $\alpha v$, de $V$ con las siguiente propiedades: 

\begin{enumerate}
  \item $\alpha (v+w)=\alpha v+\alpha w$
  \item $(\alpha+\beta )v=\alpha v+\beta v$
  \item $\alpha(\beta v)=(\alpha\beta)v$
  \item $1v=v$
\end{enumerate}

Una opercai\`on denominada multiplicaci\`on por escalares que asocia a cada escalar $c\in F  $, un vector $c \alpha \in V$ de manera que:

\begin{enumerate}[a]
	\item $(c1c2)\alpha=c1(c2\alpha)$
	\item $c(\alpha+\beta)=c\alpha+c\beta \forall c\inF \alpha,\beta\inV$
	\item $(c_1+c_2) \alpha = c_1  \alpha + c_2 \alpha \forall c_1 ,c_2 \inF \alpha \in V$
	\item $1\alpha = \alpha \forall \alpha \in V$	
\end{enumerate}
		
		\begin{definicon}
		Sea F un campo & sea $n \in N$
		
		\begin{equation}
		F^n =\{ (x_1 , \dots \dots , x_n) \ | x_1 \in F }
		\end{equation}