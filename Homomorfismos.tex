\section{Definicion de Homomorfismos } 
	\begin{definicion}
	Una aplicaion $\varepsilon :G\rightarrow \bar{G} $ $G$ es un grupo con operacion ($\bullet$ ) \& $\bar{G} $ es un grupo con operacion ($\blacksquare $) \\

	Se dice que es un homomorfismo si para $a,b\in  G$ cuales quiera se tiene que:
	  \begin{equation}
	    \varepsilon (a\bullet b)=\varepsilon (a)\blacksquare \varepsilon (b) 
	  \end{equation}
	\end{definicion} 
	
	\begin{ejemplo}
	$G=\mathbb{R} ^{+}\diagdown \{ 0\} $ 
	bajo la multiplicacion \& sea
	 $\bar{G}=\mathbb{R}   $ 
	 bajo la adicion definimos $\varepsilon :G \rightarrow \bar{G} $
	  como
	   $ \varepsilon :\mathbb{R} ^{+\diagdown \{ 0\} } \Rightarrow \mathbb{R} \rightarrow ln(r)$ 
	   sean 
	   $r_{1}r_{2}\in\mathbb{R} ^{+\diagdown \{ 0\} }$ 
	   t.q.
	    $\varepsilon (r_{1}\bullet r_{2})=ln(r_{1}r_{2})=ln(r_{1})+ln(r_{2})=\varepsilon (r_{1})+\varepsilon (r_{1})$ 
	    por tanto $\varepsilon $ es un homomorfismo.
	    \end{ejemplo}
	  \begin{lema}
	  Supongamos que $G$ es un grupo \&  que $N$ es un subgrupo de $G$ \\
	  Definamos la siguiente aplicaci\`on 
	  \begin{equation}
	    \varepsilon :G\rightarrow G\diagdown N \nonumber
	  \end{equation}
	  entonces $\varepsilon $ es 
	  \begin{equation}
	    x\rightarrow N_{x}
	  \end{equation}
	  un Homomorfismo
	  \end{lema}
	  \begin{definicion}
	  Un homomorfismo
      $\varepsilon :G\rightarrow \acute{G} $     
	  se dice que :
	  \begin{enumerate}[a]
	    \item Monorfismo si es 1-1 (inyectiva)
	    \item Epimorfismo si es suprayectiva
	    \item Isomorfismo si es biyectiva
	  \end{enumerate}
	  \end{definicion}
	  \begin{definicion}
	  Si $\varepsilon :G\rightarrow \acute{G} $ es un isomorfismo, decimos que  $G$ \& $\acute{G} $ son isomorfos \& escribimos 
	  \begin{equation}
	    G\cong \acute{G}  
	  \end{equation}
	  \end{definicion}
	\begin{proposicion}
	Si $\varepsilon :G\rightarrow \acute{G} $ es un homomorfismo entonces 
	\begin{equation}
	  Im\varepsilon < \acute{G} 
	\end{equation}
	dondela
	\begin{equation}
    Im\varepsilon =\{ y\in \acute{G}| \varepsilon _{(x)}=y,x\in G\} \subset G 	
	\end{equation}
	\end{proposicion}
	\begin{proof}
    \begin{enumerate}[i]
      \item Sean 
      \begin{equation}
        y_{1},y_{2}\in Im\varepsilon \Rightarrow y_{1}=\varepsilon (x_{1}) \& y_{2}=\varepsilon (x_{2})\in \acute{G} \nonumber
      \end{equation}
      \begin{equation}
      x_{1},x_{2}\in G \nonumber
      \end{equation}
      \begin{equation}
         y_{1},y_{2}=\varepsilon x_{1}\varepsilon x_{2}=\varepsilon (x_{1}x_{2} \Rightarrow y_{1},y_{2}\in Im\varepsilon  \nonumber
      \end{equation}
    \end{enumerate}
	\end{proof}
	
