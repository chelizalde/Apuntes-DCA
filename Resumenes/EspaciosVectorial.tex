%--------------------------------------
%       CONFIGURACIONES
%--------------------------------------
\documentclass{tufte-handout}
\usepackage[spanish,es-noquoting]{babel}
\usepackage[utf8]{inputenc}
\usepackage{amsmath}
\usepackage{amsthm}
\usepackage{amssymb}
\usepackage{graphicx}
\usepackage{bbm}
\usepackage{enumerate}
\usepackage{tikz}
\usepackage{todonotes}
\usepackage{xfrac}


\usetikzlibrary{shapes, arrows, decorations.markings}



%--------------------------------------
%       definiciones
%--------------------------------------

\newtheorem{definicion}{Definición}

\newtheorem{lema}{Lema}

\newtheorem{teorema}{Teorema}

\newtheorem{nota}{Nota}

\newtheorem{corolario}{Corolario}
 
\newtheorem{proposicion}{Proposición}

\DeclareMathOperator{\rango}{rango}

\newcommand{\tarea}[2][]
    {\todo[linecolor=black!50!white, backgroundcolor=black!30!white, bordercolor=black!30!white, #1]{#2}}

\newcommand{\faltante}[2][]
    {\todo[linecolor=red!50!white, backgroundcolor=red!30!white, bordercolor=red!30!white, #1]{#2}}

\title{Espacios Vectoriales}
\author{Christian}

\begin{document}
		\maketitle
		\tableofcontents
    	

\begin{definicion}
		
			Un espacio vectorial consta de lo siguiente 
			\end{definicion}
		\begin{enumerate}
				\item Un campo $\mathbb{F}$ de escalares
				\item Un conjunto no vacio de objetos denominados vectores 
				\item Una operaci\'on de nominada suma o adici\`on que asocia a cada par de vectores $\alpha , \beta \in V $ un vector $\alpha + \beta \in V $ llamado suma de $\alpha$ y $\beta$ que cumple lo siguiente.
		\end{enumerate}
		

	\begin{definicion}
		 Un espacio vectorial, $\mathcal{B}$ , es un conjunto de elementos $x,y,z,etc.,$ llamados vectores satisfaciendo  los siguientes axiomas.
	\end{definicion}
	 \begin{enumerate}[a]
	 	\item  Para cadapar $x$ y $y$, de vectores en $\mathcal{B}$ corresponde un vector $z$, llamada la suma de $x$ y $y$, $z=x+y$, de manera que \begin{enumerate}
	 		\item La adci\`on es conmutativa, $x+y=y+x$;
	 		\item La adici\`on es asociativa, $x+(y+z)=(x+y)+z$;
	 		\item Exite en $\mathcal{B}$ un unico vector 0 (llamado el origen) de tal manera que para todo $x$ en $\mathcal{B},x+0=0$;
	 		\item Para cada $x\in \mathcal{B}$ corresponde a un unico vector, denotado por $-x$, con  la propiedad $x+(-x)=0$ 
	 	\end{enumerate}  
	 \end{enumerate}

		\begin{definicion}
	Se dice $\beta \in V $ es una combinacion lineal de Vectores $\alpha_1 , \dots ,\alpha_n $ si existen $C_1 , \dots , C_n \in \mathbb{F}$ tales que
	 \end{definicion}
	   \begin{equation}
	   			\beta = \sum \limits_{i=1}^n \nonumber
	   \end{equation}

	\begin{definicion}
	  Un subespacio de V en $\mathbb{F}$  es un subconjunto W de V que con las operaciones heredadas de V, es el mismo un e.v. sobre $\mathbb{F}$ 
\end{definicion}

\emph{NOTA Si V es un e.v., V y $\{ \vec{0} \}$ se denomina los subespacios triviales de V. } 

\begin{proposicion}
	 Un subconjunto no vac\`io W de V es un subespacio vectorial ssi W es cerrado con respecto a las operaciones de V
\end{proposicion}

\begin{definicion}
	Sean $\alpha_1 , \dots , \alpha_n$ en V y $\mathcal{L} (\alpha_1 , \dots , \alpha_n)= \{ B | B $ es la combianci\`on lineal de $\alpha_1 , \dots , \alpha_n \}$ esto es un s.e.v. de V y se llama subespacio generado por $\alpha_i \ \ 1 \leq i \leq k$, o bien se dice que $\alpha_1, \dots , \alpha_k$ generan a $\mathcal{L}(\alpha_1, \dots , \alpha_k)$ 
\end{definicion}
					\begin{proposicion}
						La interseci\`on de cualquier colecci\`on de subescaios de V es un subespacio de V
					\end{proposicion}

			\emph{OBSERVACI\`ON La uni\`on de subespacios no necesariamente es s.e.v.} 


		\begin{definicion}
			Sean S,T subespacios de V, definimos la suma de S y T como 
		\end{definicion}
			\begin{equation}
				S+T = \{ s+t | s \in S , t \in T \} \nonumber
			\end{equation}

		\begin{proposicion}
				Si S y T son subesapcios de V, entonces S$+$T es s.e.v. de V
		\end{proposicion}

			\begin{definicion}
				Si S y T son s.e.v. de  V tales que S$+$T son s.e.v. de V tales que $S+T = V $ y $S \cap T = \{ 0 \}$ decimos que Ves la suma directa de S y  T y lo denotamos como sigue: 
			\end{definicion}
				\begin{equation*}
					V = S \oplus T 
				\end{equation*}

					\begin{proposicion}
						Si $V= S \oplus T$ entoces $\forall \alpha \in V \exists n$ unicos s,t tales que 
					\end{proposicion}
							\begin{equation*}
								\alpha = s + t
							\end{equation*}
						\begin{definicion}
										Los vectores $\alpha_1 , \dots , \alpha_k$ se dicen lienalmente  independientes si $\exists n $ escalares $a_1 , \dots , a_k$ no todos ceros tales que 		
											\end{definicion}					
												\begin{equation*}
													\sum \limits_{i=1}^k \alpha_i a_i = 0
												\end{equation*}
						\begin{definicion}
							Un conjunto A de vectores se dice l.i. si cualquier subconjunto finito de A es l.i. 
						\end{definicion}
						\begin{definicion}
							Un conjunto A de vectores de dice l.d. si existe un subconjunto finito de A no vac\`io que sea l.d.
						\end{definicion}
\end{document}
