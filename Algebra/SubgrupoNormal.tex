\newpage
\section{Subgrupo Normal}
	\begin{definicion}
		Un grupo $N$ de $G$ se dice que es un Subgrupo Normal de $G$ denotado por:

		\begin{equation}
			N \triangle G \nonumber
		\end{equation}
		
		si $\forall g \in G$ y $\forall n \in N$, se tiene que:

		\begin{equation}
			g n g^{-1} \in N
		\end{equation}
	\end{definicion}

	\begin{lema}
		$N$ es un subgrupo de G, si y solo si:
		\begin{equation}
			g N g^{-1} = N \quad \forall g \in G
		\end{equation}
	\end{lema}

	\begin{proof} 
		\begin{enumerate}[i]
			\item Si $g N g^{-1} = N \quad \forall g \in G$, entonces en particular:
			
			\begin{equation}
				g N g^{-1} \subseteq N \nonumber
			\end{equation}

			por lo que $g N g^{-1} \in N \quad \forall n \in N$, por lo tanto:

			\begin{equation}
				N \triangle G \nonumber
			\end{equation}

			\item Si $N$ es un subgrupo normal de $G$, entonces:

			\begin{equation}
				g N g^{-1} \in N \nonumber
			\end{equation}

			Si $g \in G \quad \forall n \in N$, entonces $g N g^{-1} \subseteq N$.

			Por otro lado $g^{-1} N g = g^{-1} N (g^{-1})^{-1} \subseteq N$, ademas:

			\begin{equation}
				N = e N e = g \left( g^{-1} N g \right) g^{-1} = g N g^{-1} \nonumber
			\end{equation}

			por lo tanto:

			\begin{equation}
				g N g^{-1} = N \nonumber
			\end{equation}
		\end{enumerate}
	\end{proof}

	\begin{lema}
		El subgrupo $N$ de $G$ es un subgrupo normal de $G$ ($N \triangle G$), si y solo si, toda clase lateral izquierda de $N$ en $G$ es una clase lateral derecha de $N$ en $G$.
	\end{lema}

	\begin{proof}
		Sea $aH = \left\{ ah | h \in H \right\}$ la clase lateral izquierda.
		\begin{enumerate}[i]
			\item Si $N$ es un subgrupo normal de $G \quad \forall g \in G \quad \forall n \in N$
			
			\begin{equation}
				g N g^{-1} = N \nonumber
			\end{equation}

			entonces, podemos hacer lo siguiente:

			\begin{equation}
				g N = g N e = g N \left( g^{-1} g \right) = \left( g N g^{-1} \right) g = N g \nonumber
			\end{equation}

			entonces, toda clase lateral izquierda coincide con la clase lateral derecha.

			\item Ahora supongamos que las clases laterales coinciden, entonces:

			\begin{equation}
				g N g^{-1} = \left( g N \right) g^{-1} = N g g^{-1} = N \nonumber
			\end{equation}

			por lo que podemos concluir que se trata de un subgrupo normal.
		\end{enumerate}
	\end{proof}

	\begin{definicion}
		Denotaremos $G/N$ al conjunto de las clases laterales derechas de $N$ en $G$

		\begin{equation}
			G/N = \left\{ Na | a \in G \right\}
		\end{equation}
	\end{definicion}

	\begin{teorema}
		Si $G$ es un grupo y $N$ es un subgrupo normal de $G$, entonces $G/N$ es tambien un grupo y se le denomina grupo cociente.
	\end{teorema}

	\begin{proof}
		\begin{enumerate}[i]
			\item Cerradura
			\todo{Prueba asignada a tarea}

			\item Asociatividad
			\todo{Prueba asignada a tarea}

			\item Identidad
			\begin{equation}
				N = N e
			\end{equation}

			Verificamos para un elemento $x \in G/N \implies x = N a$, $a\in G$

			\begin{equation}
				x N = N a N = N N a = N a = x \nonumber
			\end{equation}

			\begin{equation}
				N x = N N a = N a = x
			\end{equation}
			\item Inverso
			Sea $x \in G/N$ y sea $N a^{-1} \in G/N$, por verificar que $x^{-1} = N a^{-1}$, es el inverso de $x = N a$

			\begin{eqnarray}
				x x^{-1} = N a N a^{-1} = N N a a^{-1} = N e = N \\
				x^{-1} x = N a^{-1} N a = N N a^{-1} a = N e = N
			\end{eqnarray}

			por lo tanto $N a^{-1} = x^{-1}$ es el inverso de $x$, por lo que podemos concluir que $G/N$ es un grupo.

		\end{enumerate}
	\end{proof}
