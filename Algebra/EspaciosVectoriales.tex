\newpage 
\section{Espacios Vectoriales}

	\begin{definicion}

Un conjunto no vaci\`o $V$  se dice que es un espacio vectorial sobre un campo $F$ si $V$ es un grupo abeliano respecto a una operacai\`on que se denota por $+$, y si para todo $\alpha \in F$, $v\in V$ est\`a definido un elemento, escrito como $\alpha v$, de $V$ con las siguiente propiedades: 
\end{definicion}
 \begin{enumerate}
  \item $\alpha (v+w)=\alpha v+\alpha w$
  \item $(\alpha+\beta )v=\alpha v+\beta v$
  \item $\alpha(\beta v)=(\alpha\beta)v$
  \item $1v=v$
\end{enumerate}

Una opercai\`on denominada multiplicaci\`on por escalares que asocia a cada escalar $c\in F  $, un vector $c \alpha \in V$ de manera que:

\begin{enumerate}[a]
	\item $( c_1 c_2 ) \alpha = c1 ( c2 \alpha )$
	\item $c ( \alpha + \beta ) = c \alpha + c \beta \forall c \in F \alpha,\beta \in V$
	\item $(c_1 + c_2 ) \alpha = c_1  \alpha + c_2 \alpha \forall c_1 , c_2 \in F \alpha \in V$
	\item $ 1 \alpha = \alpha \forall \alpha \in V$	
\end{enumerate}
		
		\begin{definicion}
		Sea F un campo \& sea $n \in N$
		\end{definicion}		
		
		\begin{equation}
		F^n = \{ ( x_1 , \dots  , x_n) \ | x_1 \in F \} \ 
		Si  \ c \alpha = ( x_1 , \dots , x_n ) \ , \ \beta = ( y_1 , \dots , y_n )		
		\end{equation}
		
		Decimos 
		
		\begin{equation}
		\alpha + \beta = ( x_1 + y_1 , \dots , x_n + y_n  ) \ \&  \ c\alpha ( c x_1 ,c x_2 , 		\dots , c x_n ) \ \forall \ c \in F \nonumber
		\end{equation}				
		
		\begin{proposicion} Sea $V$ un espacio vectorial sobre $F$ entoces $\forall \ \alpha \in V$ se tiene que  
		
		\end{proposicion}
		 
		 \begin{equation}
		 0 \alpha = \overrightarrow{0} \nonumber
		 \end{equation}
		 
		 \begin{equation}
		 0 \alpha ( 0 + 0 ) \alpha = 0 \alpha + 0 \alpha \
		 - 0 \alpha + 0 \alpha = - 0 \alpha + 0 \nonumber
		 \end{equation}
			\\
		\begin{equation}
		0 = 0 \alpha \nonumber
		\end{equation}		
	\\	
		
	\begin{definicion}
	Se dice que $\beta \in V $ es una combinaci\`on lineal de vetores $\alpha , \dots , \alpha_n$ exiten $c_1, \dots , c_n \in F$ tales que
			\begin{equation}
				\beta = \sum\limits{i=1}{n} ci \ di \nonumber
			\end{equation}

	\end{definicion}

	\begin{definicion}	
Un subespacio de un espacio vectorial $V$ sobre $F$ es un subconjunto $W$ de $V$ que con las operaciones heredadas de $W$ es el mismo espacio vectorial sobre $F$.
		
	\end{definicion}	
\emph{Nota : si $V$ es un e.v. , $V=\{ \vec{0} \}$ se denomina subespacio triviales de $V$} 

\begin{proposicion}
 	Un subconjuto no vacio $W$ de $V$ es un subespacio vectorial ssi $W$ es cerrado con respectoa los operaciones de $V$
 \end{proposicion} 

 $\rightarrow$ Si $W$ es s.e.v. de $V$ por defenici\`'on es e.v. y sus operaciones son cerradas.
 $\leftarrow$ 

 
 
 \begin{definicion}
 	Sea $\alpha_1 \dots \alpha_k$ en $V$ \& $\mathcal{L}(\alpha_1 \dots \alpha_k)=\{ \beta	| \beta$ es combinaci\`on lineal de $\alpha_1 \ \dots \ \alpha_k$ esto es un s.e.v. de $V$ \& se llama subespacio generado por $\alpha_i \leq i = k$ o bien se dice que $\alpha_1 \dots \alpha_k$ genera a $\mathcal{L}$
 \end{definicion}

En general \\ Si $A \neq 0$	\& si $ A \subset V $ entonces $\mathcal{L} (A)= \{ \beta | \beta $ es combinaci\`on lineal de los elemntos de $A \ \}$
\begin{proposicion}
			La intersecion de cualquier coleci\`on de subespacio de $V$ es un subespacio de $V$	 	 										
	 	 									\end{proposicion}	 	 									
\begin{proof}
Sea $ \{ Wa \| \alpha \in I \} $ \& sea $ W \triangleq \bigcap W $
\end{proof}