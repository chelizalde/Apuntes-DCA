%--------------------------------------
%       CONFIGURACIONES
%--------------------------------------
\documentclass{tufte-book}
\usepackage[spanish,es-noquoting]{babel}
\usepackage[utf8]{inputenc}
\usepackage{amsmath}
\usepackage{amsthm}
\usepackage{amssymb}
\usepackage{graphicx}
\usepackage{bbm}
\usepackage{enumerate}
\usepackage{tikz}
\usepackage{todonotes}

\usetikzlibrary{shapes, arrows, decorations.markings}

\setcounter{secnumdepth}{0}
\numberwithin{equation}{chapter}

\newtheorem{definicion}{Definición}
\newtheorem{lema}{Lema}
\newtheorem{teorema}{Teorema}
\newtheorem{ejemplo}{Ejemplo}
\newtheorem{proposicion}{Proposición}
%--------------------------------------
%       TITULO
%--------------------------------------
\title{Matemáticas \\ Cinvestav}
\author{Generación 2014}
%--------------------------------------
%       INICIO DEL DOCUMENTO
%--------------------------------------
\begin{document}
  \maketitle
  \tableofcontents
%--------------------------------------
%       ALGEBRA ABSTRACTA
%--------------------------------------
  \part{Algebra abstracta}
    \chapter{Grupos}
      \section{Definici\`{o}n de grupo}
\paragraph{Definici\`{o}}
Un conjunto no vacio G en el que esta definida una operacion $*$ tal que va a mapear el producto cartesiano y los va amandar.\[ *:G\times G\rightarrow G  \].
\[ (a,b)\rightarrow (a*b) \]
\paragraph{Propiedad}
\begin{enumerate}
  \item $a*b \in G \forall a,b \in G$
  \item $a*b(b*c)=(a*b)*c \forall a,b,c \in G$
  \item $\exists  e \in G \therefore a*e=e*a=a\forall e\in G$ $"e"$ se le llama identidad o identidad de a
  \end{enumerate}
  \subparagraph{Ejemplo}
  \begin{enumerate}
    \item $\mathbb{Z}$ 
    \item Los racionales $ \mathbb{Q} $ con la suma
    \item $\mathbb{Q} ^{*} =  \mathbb{Q} \{0\} $ con la multiplicacion
    \item $G=\{e\} $ con la opercaion $e*e=e\in G$
    \item
    \item El conjunto de Matrices $G(n,\mathbb{R})$ es un grupo NO CONMUTATIVO 
    $A,b\in G(n,\mathbb{R} )$
    \item Son las matrices  
  \end{enumerate}

      \newpage
\section{Grupos Abelianos}
  \paragraph{Definicion}
  Se dice que un grupo G es \underline{Abeliano} si solo si $a*b=b*a$
  \paragraph{Ejemplo}
    El conjunto $\mathbb{Z} \diagup \mathbb{Z} _{n}$ (clase de equivalencia)
  \paragraph{Ejercicios}
    \begin{enumerate}
      \item Considere a $\mathbb{Z}$ con el producto usual Es $\mathbb{Z}$ un grupo?
      \item Considere a $\mathbb{Z}^{*}(incluye 0)$ con el producto usual es $\mathbb{Z} ^{*}$?
      \item Sea $G=\mathbb{R} \diagdown \{ 0\} $ si definimos $a\times b=a^{2}b$ G es un Grupo?
    \end{enumerate}
 \paragraph{Definiciones}
   Orden de un grupo es el numero de elementos que tiene dicho Grupo y se denota $|G| $ 
  Un Grupo G sera finito si tiene elementos finitos de elementos sea infinito
  \paragraph{Ejemplos}
\begin{description} 
  \item[Proposicion] Si G es un grupo entonces
  \begin{enumerate}
    \item El elemento identidad es uinico
    \item $\forall a\in G a^{-1}$ es unico
    \item $\forall a,b \in G(ab)^{-1}=b^{-1}a^{-1}$
    \item En general $(a_{1}\cdot a_{2}\cdot  \dots a_{n}) ^{-1} = (a_{n}^{-1}\cdot a_{n-1}^{-1}\cdot \dots a_{2}^{-1}\cdot a_{1}^{-1} ) \forall  a\in G $ 
  \end{enumerate}
  \item[Proposicion] Sea G un grupo $\forall a,b,c\in G$
  \begin{enumerate}
    \item $ab=ac \Rightarrow b=c$
    \item $ba=ca \Rightarrow b=c$
  \end{enumerate}
\end{description}
\paragraph{Verificacion}
\begin{enumerate}
  \item $b=eb=(aa^{-1})b=a^{-1}(ab)=a^{-1}(ac)=(a^{-1}a)c=ec=c$
  \item $b=be=b(aa^{-1})=(ba)a^{-1}=(ca)a^{-1}=c(aa^{-1})=ce=c$
\end{enumerate}

      \newpage
\section{Subgrupo}
\paragraph{Definici\`on}
  Conjunto no vac\`io H de un grupo G, se llama Subgrupo si H mismo forma un grupo respecto a la operaci\`{o} de G.
  Cuando H es subgrupo de G se denota $H< G$  \`{o} $G> H$.
  \paragraph{Observaci\`on}
  Todo grupo tiene aut\`omaticamente dos subgrupos tribiales  $G \& \{ e \} $
    \paragraph{Propoci\`on}
  Un subconjunto no vaio $H\subset G$ es un subgrupo de G ssi H es cerrado respecto a la operaci\`{o}n G \& $a\in H\Rightarrow a^{-1}\in a^{-1}\in H$ 
  \\ $\Rightarrow$ \paragraph{Necesidad} Como H es un subgrupo de G, H es un grupo y tiene inversa
  \\  $\Leftarrow$ \paragraph{Suficiencia} H es cerrado, no vacio \& y el  inverso esta en $H\forall a\in H=aa^{-1}(H es cerrado)\Rightarrow aa^{-1}=e\in H$ 
  \\
  Ademas para $a,b,c\in H$  $a(bc)=(ab)c$ $H\in G$

  \paragraph{Ejercicio} Sea $G=\mathbb{Z} $ con la  suma usual \& sea H el conjunto de enteros pares.
  \\ \[H=\{2n\diagup n\in \mathbb{Z} \}   \] H es un subgrupo?
  \paragraph{Sean a,b$\in$H} 
  $a=2q$ , $q\in \mathbb{Z} $  $b=2\acute{q}$ $\acute{q}\in \mathbb{Z}  $
  \\
  $a+b=2q+2q=2(q+q')=2q'' $
  

      \newpage
\section{Subgrupo Normal}
	\begin{definicion}
		Un grupo $N$ de $G$ se dice que es un Subgrupo Normal de $G$ denotado por:

		\begin{equation}
			N \triangle G \nonumber
		\end{equation}
		
		si $\forall g \in G$ y $\forall n \in N$, se tiene que:

		\begin{equation}
			g n g^{-1} \in N
		\end{equation}
	\end{definicion}

	\begin{lema}
		$N$ es un subgrupo de G, si y solo si:
		\begin{equation}
			g N g^{-1} = N \quad \forall g \in G
		\end{equation}
	\end{lema}

	\begin{proof} 
		\begin{enumerate}[i]
			\item Si $g N g^{-1} = N \quad \forall g \in G$, entonces en particular:
			
			\begin{equation}
				g N g^{-1} \subseteq N \nonumber
			\end{equation}

			por lo que $g N g^{-1} \in N \quad \forall n \in N$, por lo tanto:

			\begin{equation}
				N \triangle G \nonumber
			\end{equation}

			\item Si $N$ es un subgrupo normal de $G$, entonces:

			\begin{equation}
				g N g^{-1} \in N \nonumber
			\end{equation}

			Si $g \in G \quad \forall n \in N$, entonces $g N g^{-1} \subseteq N$.

			Por otro lado $g^{-1} N g = g^{-1} N (g^{-1})^{-1} \subseteq N$, ademas:

			\begin{equation}
				N = e N e = g \left( g^{-1} N g \right) g^{-1} = g N g^{-1} \nonumber
			\end{equation}

			por lo tanto:

			\begin{equation}
				g N g^{-1} = N \nonumber
			\end{equation}
		\end{enumerate}
	\end{proof}

	\begin{lema}
		El subgrupo $N$ de $G$ es un subgrupo normal de $G$ ($N \triangle G$), si y solo si, toda clase lateral izquierda de $N$ en $G$ es una clase lateral derecha de $N$ en $G$.
	\end{lema}

	\begin{proof}
		Sea $aH = \left\{ ah | h \in H \right\}$ la clase lateral izquierda.
		\begin{enumerate}[i]
			\item Si $N$ es un subgrupo normal de $G \quad \forall g \in G \quad \forall n \in N$
			
			\begin{equation}
				g N g^{-1} = N \nonumber
			\end{equation}

			entonces, podemos hacer lo siguiente:

			\begin{equation}
				g N = g N e = g N \left( g^{-1} g \right) = \left( g N g^{-1} \right) g = N g \nonumber
			\end{equation}

			entonces, toda clase lateral izquierda coincide con la clase lateral derecha.

			\item Ahora supongamos que las clases laterales coinciden, entonces:

			\begin{equation}
				g N g^{-1} = \left( g N \right) g^{-1} = N g g^{-1} = N \nonumber
			\end{equation}

			por lo que podemos concluir que se trata de un subgrupo normal.
		\end{enumerate}
	\end{proof}

	\begin{definicion}
		Denotaremos $G/N$ al conjunto de las clases laterales derechas de $N$ en $G$

		\begin{equation}
			G/N = \left\{ Na | a \in G \right\}
		\end{equation}
	\end{definicion}

	\begin{teorema}
		Si $G$ es un grupo y $N$ es un subgrupo normal de $G$, entonces $G/N$ es tambien un grupo y se le denomina grupo cociente.
	\end{teorema}

	\begin{proof}
		\begin{enumerate}[i]
			\item Cerradura
			\todo{Prueba asignada a tarea}

			\item Asociatividad
			\todo{Prueba asignada a tarea}

			\item Identidad
			\begin{equation}
				N = N e
			\end{equation}

			Verificamos para un elemento $x \in G/N \implies x = N a$, $a\in G$

			\begin{equation}
				x N = N a N = N N a = N a = x \nonumber
			\end{equation}

			\begin{equation}
				N x = N N a = N a = x
			\end{equation}
			\item Inverso
			Sea $x \in G/N$ y sea $N a^{-1} \in G/N$, por verificar que $x^{-1} = N a^{-1}$, es el inverso de $x = N a$

			\begin{eqnarray}
				x x^{-1} = N a N a^{-1} = N N a a^{-1} = N e = N \\
				x^{-1} x = N a^{-1} N a = N N a^{-1} a = N e = N
			\end{eqnarray}

			por lo tanto $N a^{-1} = x^{-1}$ es el inverso de $x$, por lo que podemos concluir que $G/N$ es un grupo.

		\end{enumerate}
	\end{proof}
    \chapter{Homorfismos de grupo}
      \newpage
\section{Definici\`{o}n}
Un mapeo $\phi$ de un grupo G en un grupo  $\bar{G} $ se dice ser un homomorfismo si para todo $a,b\in G,\phi (a,b)=\phi (a)\phi (b)$ 
  \paragraph{Proposici\`{o}n}
  Sea $\varepsilon :G\rightarrow \acute{G}  $ homomorfismo. Entonces $\varepsilon $ es un monomorfismo ssi $ker\varepsilon =\{0\} (e=0\in G)$
  \paragraph{Verificaci\`{o}n} 
  $\Leftarrow $ \\
  Supongamos que Ker$\varepsilon =\{0\}$ por verificar que $\varepsilon$ es monomorfismo \\ $\Rightarrow$   \\ Supongamos que $\varepsilon(x_{1})=\varepsilon (x_{2})$ 
    \\ 

    \chapter{Anillos}
    \chapter{Dominios}
%--------------------------------------
%       ALGEBRA LINEAL
%--------------------------------------
  \part{Algebra Lineal}
    \chapter{Espacios vectoriales}
    \newpage 
\section{Espacios Vectoriales}

	\begin{definicion}

Un conjunto no vaci\`o $V$  se dice que es un espacio vectorial sobre un campo $F$ si $V$ es un grupo abeliano respecto a una operacai\`on que se denota por $+$, y si para todo $\alpha \in F$, $v\in V$ est\`a definido un elemento, escrito como $\alpha v$, de $V$ con las siguiente propiedades: 

\begin{enumerate}
  \item $\alpha (v+w)=\alpha v+\alpha w$
  \item $(\alpha+\beta )v=\alpha v+\beta v$
  \item $\alpha(\beta v)=(\alpha\beta)v$
  \item $1v=v$
\end{enumerate}

Una opercai\`on denominada multiplicaci\`on por escalares que asocia a cada escalar $c\in F  $, un vector $c \alpha \in V$ de manera que:

\begin{enumerate}[a]
	\item $(c1c2)\alpha=c1(c2\alpha)$
	\item $c(\alpha+\beta)=c\alpha+c\beta \forall c\inF \alpha,\beta\inV$
	\item $(c_1+c_2) \alpha = c_1  \alpha + c_2 \alpha \forall c_1 ,c_2 \inF \alpha \in V$
	\item $1\alpha = \alpha \forall \alpha \in V$	
\end{enumerate}
		
		\begin{definicon}
		Sea F un campo & sea $n \in N$
		
		\begin{equation}
		F^n =\{ (x_1 , \dots \dots , x_n) \ | x_1 \in F }
		\end{equation}
    \chapter{Operadores lineales}
    \chapter{Funciones lineales}
    \chapter{Espacios duales}
    \chapter{Teorema de Caley-Hamilton}
    \chapter{Diagonalizacion}
    \chapter{Forma canónica de Jordan}
    \chapter{Vectores propios generalizados}
%--------------------------------------
%       ECUACIONES DIFERENCIALES
%--------------------------------------
  \part{Ecuaciones difereciales}
    \chapter{Resolución de ecuaciones diferenciales}
    \chapter{Existencia y unicidad de la solución de una ED}
    \chapter{Solución aproximada}
    \chapter{Relación entre soluciones aproximadas y exactas}
%--------------------------------------
%       FIN DEL DOCUMENTO
%--------------------------------------
\end{document}
