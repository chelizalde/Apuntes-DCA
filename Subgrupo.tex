\newpage
\section{Subgrupo}
\paragraph{Definici\`on}
  Conjunto no vac\`io H de un grupo G, se llama Subgrupo si H mismo forma un grupo respecto a la operaci\`{o} de G.
  Cuando H es subgrupo de G se denota $H< G$  \`{o} $G> H$.
  \paragraph{Observaci\`on}
  Todo grupo tiene aut\`omaticamente dos subgrupos tribiales  $G \& \{ e \} $
    \paragraph{Propoci\`on}
  Un subconjunto no vaio $H\subset G$ es un subgrupo de G ssi H es cerrado respecto a la operaci\`{o}n G \& $a\in H\Rightarrow a^{-1}\in a^{-1}\in H$ 
  \\ $\Rightarrow$ \paragraph{Necesidad} Como H es un subgrupo de G, H es un grupo y tiene inversa
  \\  $\Leftarrow$ \paragraph{Suficiencia} H es cerrado, no vacio \& y el  inverso esta en $H\forall a\in H=aa^{-1}(H es cerrado)\Rightarrow aa^{-1}=e\in H$ 
  \\
  Ademas para $a,b,c\in H$  $a(bc)=(ab)c$ $H\in G$

  \paragraph{Ejercicio} Sea $G=\mathbb{Z} $ con la  suma usual \& sea H el conjunto de enteros pares.
  \\ \[H=\{2n\diagup n\in \mathbb{Z} \}   \] H es un subgrupo?
  \paragraph{Sean a,b$\in$H} 
  $a=2q$ , $q\in \mathbb{Z} $  $b=2\acute{q}$ $\acute{q}\in \mathbb{Z}  $
  \\
  $a+b=2q+2q=2(q+q')=2q'' $
  
