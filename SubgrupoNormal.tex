\newpage
\section{Subgrupo Normal}
	\begin{definicion}
		Un grupo $N$ de $G$ se dice que es un Subgrupo Normal de $G$ denotado por:

		\begin{equation}
			N \triangle G \nonumber
		\end{equation}
		
		si $\forall g \in G$ y $\forall n \in N$, se tiene que:

		\begin{equation}
			g n g^{-1} \in N
		\end{equation}
	\end{definicion}

	\begin{lema}
		$N$ es un subgrupo de G, si y solo si:
		\begin{equation}
			g N g^{-1} = N \quad \forall g \in G
		\end{equation}
	\end{lema}

	\begin{proof} 
		\begin{enumerate}[i]
			\item Si $g N g^{-1} = N \quad \forall g \in G$, entonces en particular:
			
			\begin{equation}
				g N g^{-1} \subseteq N \nonumber
			\end{equation}

			por lo que $g N g^{-1} \in N \quad \forall n \in N$, por lo tanto:

			\begin{equation}
				N \triangle G \nonumber
			\end{equation}
			
			\item Si $N$ es un subgrupo normal de $G$, entonces:

			\begin{equation}
				g N g^{-1} \in N \nonumber
			\end{equation}

			Si $g \in G \quad \forall n \in N$, entonces $g N g^{-1} \subseteq N$.

			Por otro lado $g^{-1} N g = g^{-1} N (g^{-1})^{-1} \subseteq N$, ademas:

			\begin{equation}
				N = e N e = g \left( g^{-1} N g \right) g^{-1} = g N g^{-1} \nonumber
			\end{equation}

			por lo tanto:

			\begin{equation}
				g N g^{-1} = N \nonumber
			\end{equation}
		\end{enumerate}
	\end{proof}