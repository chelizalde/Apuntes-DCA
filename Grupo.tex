\section{Definici\`{o}n de grupo}
\paragraph{Definici\`{o}}
Un conjunto no vacio G en el que esta definida una operacion $*$ tal que va a mapear el producto cartesiano y los va amandar.\[ *:G\times G\rightarrow G  \].
\[ (a,b)\rightarrow (a*b) \]
\paragraph{Propiedad}
\begin{enumerate}
  \item $a*b \in G \forall a,b \in G$
  \item $a*b(b*c)=(a*b)*c \forall a,b,c \in G$
  \item $\exists  e \in G \therefore a*e=e*a=a\forall e\in G$ $"e"$ se le llama identidad o identidad de a
  \end{enumerate}
  \subparagraph{Ejemplo}
  \begin{enumerate}
    \item $\mathbb{Z}$ 
    \item Los racionales $ \mathbb{Q} $ con la suma
    \item $\mathbb{Q} ^{*} =  \mathbb{Q} \{0\} $ con la multiplicacion
    \item $G=\{e\} $ con la opercaion $e*e=e\in G$
    \item
    \item El conjunto de Matrices $G(n,\mathbb{R})$ es un grupo NO CONMUTATIVO 
    $A,b\in G(n,\mathbb{R} )$
    \item Son las matrices  
  \end{enumerate}
